\chapter{Einleitung + \LaTeX{} zum Template}
\pagenumbering{arabic} %Ab hier, werden arabische Zahlen benutzt
\setcounter{page}{1} %Mit Abschnitt 1 beginnt die Seitennummerierung neu.
\thispagestyle{empty}

Die Einleitung soll zum eigentlichen Themengebiet hinf�hren und die
Motivation f�r die Arbeit liefern. Am Schlu� der Einleitung wird
weiterhin noch eine �bersicht �ber die restliche Arbeit gegeben.\\

\LaTeX{}-Anleitung \\
Im Prinzip ist der allgemeine Gebrauch von diesem Template ganz easy.\\
Auskommentieren per \% ist zu bevorzugen damit man zur Not reparieren kann.

\begin{enumerate}		
  \item In der 1Main.tex Datei Zeile 50 Sprachen einstellen.			
  \item In der 1Main.tex Datei ab Zeile 55 alle Daten eintragen. \\ 
  Wenn nur ein Betreuer: Zeile 64 /advisortwo auskommentieren und in der styleguide.sty Datei die Zeilen 154,208,260 auskommentieren
  \item Die Quellen welche man nutzen will in die .bib-Datei einf�gen.
  \item Ab Zeile 71 beginnt nun das eigentliche Dokument bzw es werden die einzelnen Teile eingebunden. Also folgende Dateien mit den eigenen Texten bef�llen:
  	\begin{itemize}
  		\item KurzfassungDT.tex und KurzfassungEN.tex
  		\item Einleitung.tex
  		\item Hauptteil.tex
  		\item Zusammenfassung.tex
  		\item Anhaenge.tex
  		\item Abkuerzungsverzeichnis.tex
	\end{itemize}
\end{enumerate}

