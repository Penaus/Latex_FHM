\chapter{Ein Kapitel des Hauptteils}
\thispagestyle{empty}% no page number in chapter title page
\section{Inhalte}
Im Hauptteil werden aufbauend auf einer State-of-the-art-Diskussion
(Literaturrecherche) die Ergebnisse der Arbeit im Detail vorgestellt.
Dabei sollen auch die Schritte des durchgef�hrten Arbeitsprozesses
darstellt werden.  Dazu kann der Hauptteil in mehrere Kapitel
unterteilt werden.

Einleitung und Hauptteil sollen eine in sich geschlossene Abhandlung
darstellen. Der Leser der Arbeit soll ohne zus�tzliche Literatur in
der Lage sein, die Arbeit im Zusammenhang zu verstehen.

\section{Beispiel f�r eine Abbildung}
\begin{figure}[h!]
  \begin{center}
        \includegraphics[width=4cm]{Hochschule_Muenchen_Logo}
    \caption{Beispiel f�r eine Beschriftung.}
    \label{fig:ToUseWithReference}
  \end{center}
\end{figure}

Durch die \texttt{\bslash label} kann auf die Bilder mit
\texttt{\bslash ref} verwiesen werden
(z.B.~Abbildung~\ref{fig:ToUseWithReference}).
\begin{verbatim}
(z.B.~Abbildung~\ref{fig:ToUseWithReference}).
\end{verbatim}
\clearpage


\section{Schrifttypen}
Als Schrifttyp wird Arial oder Roman empfohlen. Bitte beachten, da�
Gr��en und Einheiten eine eigene Schreibweise haben:
\begin{description}
\item[Kursivschrift:] physikalische Gr��en (z.B.~$U$ f�r Spannung),
  Variablen~(z.B.~$x$), sowie Funktions- und Operatorzeichen, deren
  Bedeutung frei gew�hlt werden kann (z.B.~$f(x)$)
\item[Steilschrift:] Einheiten und ihre Vors�tze (z.B.~kg, pF),
  Zahlen, Funktions- und Operatorzeichen mit feststehender Bedeutung
  (z.B.~sin, lg)
\end{description}
Die verschiedenen Schriften k\"onnen auch einfach per Befehl f\"ur das jeweiligen Wort aktiviert werden.
\begin{verbatim} 
Ich bin ein \textbf{Test} um die \underline{Besonderheiten} in 
\textit{Latex} zu zeigen.
\end{verbatim} 
Ich bin ein \textbf{Test} um die \underline{Besonderheiten} in \textit{Latex} zu zeigen.

\section{Akronyme}
Man kann auch ganz leicht mit Abk\"urzungen und dem Abk\"urzungenverzeichniss zu arbeiten.\\
LaTex holt sich die Infos dazu aus dem Abk\"urzungenverzeichniss im Anhang.
Hierf\"ur einfach { \begin{verbatim} 	\ac{Bash} \end{verbatim} } schreiben. 
Beim ersten Gebrauch wird die ausgeschrieben Abk\"urzung genutzt. \\
Anschlie\ss end immer das K\"urzel.
\begin{tabbing}
Beim ersten Mal so: \=  \ac{Bash}  \\
Beim zweiten Mal   \>   \ac{Bash}. \\
\end{tabbing}
\clearpage


\section{Zitate}
Zitate aus B\"uchern und Texten kann man auch recht leicht einf\"ugen.
Zuvor muss aber eine Datei .bib mit den genutzten Infos eingebunden werden.\\
{ \begin{verbatim} 
\bibliography{Bibliothek}
\end{verbatim} } 
Per { \begin{verbatim} 
\cite{Hertzberg.2012} 
\cite[S. 15]{Hertzberg.2012} % Angabe mit Seitenzahl
\cite[S.15 Z.1-2]{Hertzberg.2012}  % Angabe mit Seite+Zeile
\end{verbatim} } 
wird ein Zitat mit Verweis auf unsere Quelle angegeben.

\cite{Hertzberg.2012} \\
\cite[S. 15]{Hertzberg.2012} \\
\cite[S.15 Z.1-2]{Hertzberg.2012} \\

\section{Zitate - Quellen}

Wie bereits zuvor angedeutet verwenden Zitate eine .bib Datei, welche man leicht mit einer Quellenverwaltung erstellen kann. Hierf�r gibt es von der Hochschule Lizenzen f�r Citavi sowie Zotero. Da ich in meiner Arbeit Citavi+Texmaker verwende, gehe ich nur hierauf ein. Eine Einarbeitung in Zotero ist aber auch leicht.
{ \begin{verbatim} 
https://bib.hm.edu/tutorials/citavi/index.de.html
\end{verbatim} } 
Der Ablauf ist recht einfach:
\begin{enumerate}
  \item Projekt in Citavi erstellen.
  \item Per Citavi-Picker bzw manuell B�cher, Artikel etc hinzuf�gen.
  \item Citavi - Extras - Optionen - Zitation - Standardeditior auf TexMaker.
  \item Citavi - Datei - Exportieren - Exportieren - Alle Titel - BibTeX - In Zwischenablage.
  \item Jetzt in die Bibliethek.bib im Ordner die Zwischenablage einf�gen.
  \item In TexMaker - Optionen - Texmaker konfigurieren - Schnelles �bersetzen
  \item PdfLatex + Bib(la)tex + PdfLatex(2x)+PDF Anzeigen.
  \item Beim n�chsten "'Schnelles �bersetzen"' wird die .bib genutzt.
\end{enumerate}


\section{Fu\ss noten}
Fu\ss noten k\"onnen auch leicht eingef\"ugt werden. Hierzu einfach
\begin{verbatim}
	\footnote{Ich bin der Fu\ss notentext}
\end{verbatim}
verwenden. Die Fu\ss note \footnote{Ich bin der Fu\ss notentext} wird dann am jeweiligen Seitenende angezeigt.

\section{Listen}
Listen kann man auch ganz leicht erstellen. Hierbei kann zwischen Listen mit Punkten und mit f�hrenden Zahlen gew�hlt werden.

\begin{minipage}[t]{0.5\textwidth}
\begin{itemize}
  \item Ein erster Punkt
  \item Ein zweiter Punkt
\end{itemize}
\end{minipage}
\begin{minipage}[t]{0.5\textwidth}
\begin{enumerate}
  \item Ein erster Punkt
  \item Ein zweiter Punkt
\end{enumerate}
\end{minipage}
\\
\begin{verbatim}
\begin{itemize}			// "enumerate" in den {}Klammern bewirken f�hrende Zahlen
  \item Ein erster Punkt
  \item Ein zweiter Punkt
\end{itemize}
\end{verbatim}
Damit die Tabellen wie oben nebeneinander sind kann der Befehl "minipage" genutzt werden. Man erstellt somit sozusagen eine Seite in der Seite.
\begin{verbatim}
\begin{minipage}[t]{0.5\textwidth}	//2 Elementen -> 0,5 Seitenbreite; 3 -> 0,3
\begin{itemize}						
  \item Ein erster Punkt
  \item Ein zweiter Punkt
\end{itemize}
\end{minipage}
\end{verbatim}
Um Unterlisten zu erstellen einfach in der Liste eine neue Liste beginnen. LaTex macht dann den Rest.

\clearpage

\section{Tabellen}
Tabellen werden mithilfe des \& Zeichens und den Befehlen tabular und hline erstellt.

\begin{minipage}[t]{0.5\textwidth}
\begin{center}
\vspace{12mm}
\begin{tabular}{|c|c|c|}
\hline
 A & B & C \\
\hline
 1 & 2 & 3 \\
\hline 
 4 & 5 & 6 \\
\hline
\end{tabular}
\end{center}
\end{minipage}
\begin{minipage}[t]{0.5\textwidth}
\begin{verbatim}
\begin{tabular}{|c|c|c|}
\hline
 A & B & C \\
\hline
 1 & 2 & 3  \\
\hline 
 4 & 5 & 6 \\
\hline
\end{tabular}
\end{verbatim}
\end{minipage}

\section{Archivierung}
F�r die Archivierung sind alle Dateien der Arbeit (auch der Vortr�ge)
dem Betreuer zur Verf�gung zu stellen.  Weiterhin soll noch ein
\BibTeX-Eintrag der Arbeit erstellt werden (die Felder in eckigen
Klammern sind dabei auszuf�llen):
\begin{verbatim}
@mastersthesis{<Nachname des Autors><Jahr>,	
  type =         {<Art der Arbeit>},
  title =        ,
  school =       {Munich University of applied Science (HM)},
  author =       {<Nachname des Autors>, <Vorname des Autors>},
  annote =       {<Nachname des Betreuers>, <Vorname des Betreuers>},
  month =        {<Monat>},
  year =         {<Jahr>},
  key =          {<Mehrere Suchschl�ssel>}
}
\end{verbatim}
