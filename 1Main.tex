\documentclass[12pt,a4paper]{report}

%------------------------------------------------------------------------
% PACKAGES:
\usepackage{lipsum}
% Lorem ipsum package
\usepackage{comment}
% Allows to comment text out like in C using '#'
\usepackage{acronym}
% Allows using acronyms
\usepackage{styleguide}
% Define typearea
% a) Use automatic:
\usepackage[BCOR1cm]{typearea}
% b) Or use fixed:
%\usepackage{geometry}
%\geometry{left=1.5cm,textwidth=18.5cm,top=1.5cm,textheight=26.5cm}
\usepackage[american,ngerman]{babel}
% Use list of tabels, etc. in table of contents:
\usepackage{tocbibind}
% German paragraph skip
\usepackage{parskip}
% Encoder:????
\usepackage[latin1]{inputenc}
%\usepackage[applemac]{inputenc}
% Use A4-paper efficiently:
\usepackage{a4wide}
% Index-generation
\usepackage{makeidx}
% Einbinden von URLs:
\usepackage{url}
% Include .eps-files (needed also for the logo):
%\usepackage{epsf}
% Special \LaTex symbols (e.g. \BibTeX):
\usepackage{doc}
% Include Graphic-files:
%\usepackage{graphics}
% Include Graphic-files:
\usepackage{graphicx}
% Include doc++ generated tex-files:
%\usepackage{docxx}
% Include PDF links
%\usepackage[pdftex, bookmarks=true]{hyperref}
%------------------------------------------------------------------------
\graphicspath{{figures/}}


%########################################################################
% CHOOSE DEFAULT LANGUAGE:
\setlang{de}
% \setlang{en}

% DATEN EINTRAGEN:
% FILL IN ACCORDINGLY:

\title{XX}
\type{B} % / M:Master /B:Bachelor / F:Forschungspraxis / I:Ingenieurspraxis
\author{XX}
\matrikelnr{XX}
\fakultaet{6} % / Nummer der Fakultät - 1...14
\street{XX}
\town{XX}
\advisor{Prof. XX}
\advisortwo{Prof. XX}
% Wenn nur ein Betreuer -> Änderung in styleguide.sty -> Zeile 154,208,260 auskommentieren
\datebegin{Datum des Arbeitsbeginns}
\dateend{Datum des Vortrags}
%########################################################################


\begin{document}%****************************************************

\makemtgtitle
% Adding the declaration of independence according to language
{\ifthenelse{\equal{\lang}{de}}
	{ \thispagestyle{plain}
\pagenumbering{gobble}
\switchlanguage{de}
% \vspace*{1cm}
\section*{Erkl\"arung} 
In Anlehnung an den \S26 ,Abs.7 der Allg. Studien- und Pr\"ufungsordnung (ASPO) der Hochschule M\"unchen vom 05.01.2018. \\
Hiermit erkl\"are ich gem\"a\ss \ der Rahmenpr\"ufungsordnung der Hochschule M\"unchen, dass ich die vorliegende Arbeit mit dem Titel 
\begin{center}
\rule{16cm}{0.4pt}\\
\vspace{2mm}
\inserttitle \\
\rule{16cm}{0.4pt}\\
\vspace{2mm}
\end{center}
selbst\"andig verfasst, noch nicht anderweitig f\"ur Pr\"ufungszwecke vorgelegt, keine anderen als die angegebenen Quellen oder Hilfsmittel benutzt sowie w\"ortliche und sinngem\"a\ss e Zitate als solche gekennzeichnet habe.

\begin{center}

\vspace{2cm}
\hspace{1cm}
\begin{tabular}{ccc}
\vspace{-0.3cm}M\"unchen, \today	&\hspace{2cm} 		& \\
\rule{6cm}{0.8pt}					&					&\rule{6cm}{0.8pt}\\
Ort, Datum							&					& Unterschrift			
\end{tabular}
\end{center}
           		


\begin{comment}
\vspace{2cm}
Diese Arbeit ist unter der Lizenz Creative Commons Attribution 3.0 Germany verf\"ugbar. Eine Kopie der Lizenz kann unter http://creativecommons.org/licenses/by/3.0/de eingesehen oder durch einen Brief an Creative Commons, 171 Second Street, Suite 300, San Francisco, California 94105, USA erfragt werden.

\begin{center}
\vspace{2cm}

\hspace{1cm}\begin{tabular}{ccc}
\vspace{-0.3cm}M\"unchen, \today	&\hspace{2cm} 		& \\
\rule{6cm}{0.8pt}					&					&\rule{6cm}{0.8pt}\\
Ort, Datum							&					& Unterschrift			
\end{tabular}
\end{center}
\end{comment} } 
		{ \thispagestyle{plain}
\pagenumbering{gobble}
\switchlanguage{en}
\vspace*{1cm}

\section*{Declaration} 
Based on the General Study and Examination Regulations ( \S26 Abs.7 "Allgemeine Studien- und Pr\"ufungsordnung (ASPO)")  of the University of Applied Sciences Munich issued 05.01.2018.\\ \\

With my signature below, I assert that the work in this thesis has been composed by myself independently and no source materials or aids other than those mentioned in the thesis have been used.

\begin{center}
\vspace{2cm}

\hspace{1cm}\begin{tabular}{ccc}
\vspace{-0.3cm}M\"unchen, \today 	&\hspace{2cm} 		& \\
\rule{6cm}{0.8pt}					&					&\rule{6cm}{0.8pt}\\
Place, Date							&					& Signature			
\end{tabular}
\end{center}
           		

\begin{comment}
\vspace{4cm}
This work is licensed under the Creative Commons Attribution 3.0 Germany License. To view a copy of the license, visit http://creativecommons.org/licenses/by/3.0/de\\

Or\\

Send a letter to Creative Commons, 171 Second Street, Suite 300, San Francisco, California 94105, USA.

\begin{center}
\vspace{2cm}

\hspace{1cm}\begin{tabular}{ccc}
\vspace{-0.3cm}M\"unchen, \today 	&\hspace{4cm} 		& \\
\rule{6cm}{0.8pt}					&					&\rule{6cm}{0.8pt}\\
Place, Date							&					& Signature	
\end{tabular}
\end{center}
\end{comment} }
% MAIN PART:
% Start numbering for pages
\pagenumbering{roman}
% German abstract:
\switchlanguage{de} % The Kurzfassung, if given, is supposed to be in German!

\thispagestyle{plain}

\section*{Kurzfassung} 
Hier folgt eine Kurzfassung der Arbeit.\\
Dabei wird ein kleiner �berblick �ber Einstieg, Verlauf sowie Fazit gegeben.


\switchlanguage{\lang} % Switch back to the docmuent's default language.
% English abstract:
\include{KurzfassungEN}
% Table of contents:
\tableofcontents 

% Introduction (Einleitung):
\chapter{Einleitung + \LaTeX{} zum Template}
\pagenumbering{arabic} %Ab hier, werden arabische Zahlen benutzt
\setcounter{page}{1} %Mit Abschnitt 1 beginnt die Seitennummerierung neu.
\thispagestyle{empty}

Die Einleitung soll zum eigentlichen Themengebiet hinf�hren und die
Motivation f�r die Arbeit liefern. Am Schlu� der Einleitung wird
weiterhin noch eine �bersicht �ber die restliche Arbeit gegeben.\\

\LaTeX{}-Anleitung \\
Im Prinzip ist der allgemeine Gebrauch von diesem Template ganz easy.\\
Auskommentieren per \% ist zu bevorzugen damit man zur Not reparieren kann.

\begin{enumerate}		
  \item In der 1Main.tex Datei Zeile 50 Sprachen einstellen.			
  \item In der 1Main.tex Datei ab Zeile 55 alle Daten eintragen. \\ 
  Wenn nur ein Betreuer: Zeile 64 /advisortwo auskommentieren und in der styleguide.sty Datei die Zeilen 154,208,260 auskommentieren
  \item Die Quellen welche man nutzen will in die .bib-Datei einf�gen.
  \item Ab Zeile 71 beginnt nun das eigentliche Dokument bzw es werden die einzelnen Teile eingebunden. Also folgende Dateien mit den eigenen Texten bef�llen:
  	\begin{itemize}
  		\item KurzfassungDT.tex und KurzfassungEN.tex
  		\item Einleitung.tex
  		\item Hauptteil.tex
  		\item Zusammenfassung.tex
  		\item Anhaenge.tex
  		\item Abkuerzungsverzeichnis.tex
	\end{itemize}
\end{enumerate}



% Text Body (Hauptteil)
% Could have multiple chaper-files, e.g.:
\chapter{Ein Kapitel des Hauptteils}
\thispagestyle{empty}% no page number in chapter title page
\section{Inhalte}
Im Hauptteil werden aufbauend auf einer State-of-the-art-Diskussion
(Literaturrecherche) die Ergebnisse der Arbeit im Detail vorgestellt.
Dabei sollen auch die Schritte des durchgef�hrten Arbeitsprozesses
darstellt werden.  Dazu kann der Hauptteil in mehrere Kapitel
unterteilt werden.

Einleitung und Hauptteil sollen eine in sich geschlossene Abhandlung
darstellen. Der Leser der Arbeit soll ohne zus�tzliche Literatur in
der Lage sein, die Arbeit im Zusammenhang zu verstehen.

\section{Beispiel f�r eine Abbildung}
\begin{figure}[h!]
  \begin{center}
        \includegraphics[width=4cm]{Hochschule_Muenchen_Logo}
    \caption{Beispiel f�r eine Beschriftung.}
    \label{fig:ToUseWithReference}
  \end{center}
\end{figure}

Durch die \texttt{\bslash label} kann auf die Bilder mit
\texttt{\bslash ref} verwiesen werden
(z.B.~Abbildung~\ref{fig:ToUseWithReference}).
\begin{verbatim}
(z.B.~Abbildung~\ref{fig:ToUseWithReference}).
\end{verbatim}
\clearpage


\section{Schrifttypen}
Als Schrifttyp wird Arial oder Roman empfohlen. Bitte beachten, da�
Gr��en und Einheiten eine eigene Schreibweise haben:
\begin{description}
\item[Kursivschrift:] physikalische Gr��en (z.B.~$U$ f�r Spannung),
  Variablen~(z.B.~$x$), sowie Funktions- und Operatorzeichen, deren
  Bedeutung frei gew�hlt werden kann (z.B.~$f(x)$)
\item[Steilschrift:] Einheiten und ihre Vors�tze (z.B.~kg, pF),
  Zahlen, Funktions- und Operatorzeichen mit feststehender Bedeutung
  (z.B.~sin, lg)
\end{description}
Die verschiedenen Schriften k\"onnen auch einfach per Befehl f\"ur das jeweiligen Wort aktiviert werden.
\begin{verbatim} 
Ich bin ein \textbf{Test} um die \underline{Besonderheiten} in 
\textit{Latex} zu zeigen.
\end{verbatim} 
Ich bin ein \textbf{Test} um die \underline{Besonderheiten} in \textit{Latex} zu zeigen.

\section{Akronyme}
Man kann auch ganz leicht mit Abk\"urzungen und dem Abk\"urzungenverzeichniss zu arbeiten.\\
LaTex holt sich die Infos dazu aus dem Abk\"urzungenverzeichniss im Anhang.
Hierf\"ur einfach { \begin{verbatim} 	\ac{Bash} \end{verbatim} } schreiben. 
Beim ersten Gebrauch wird die ausgeschrieben Abk\"urzung genutzt. \\
Anschlie\ss end immer das K\"urzel.
\begin{tabbing}
Beim ersten Mal so: \=  \ac{Bash}  \\
Beim zweiten Mal   \>   \ac{Bash}. \\
\end{tabbing}
\clearpage


\section{Zitate}
Zitate aus B\"uchern und Texten kann man auch recht leicht einf\"ugen.
Zuvor muss aber eine Datei .bib mit den genutzten Infos eingebunden werden.\\
{ \begin{verbatim} 
\bibliography{Bibliothek}
\end{verbatim} } 
Per { \begin{verbatim} 
\cite{Hertzberg.2012} 
\cite[S. 15]{Hertzberg.2012} % Angabe mit Seitenzahl
\cite[S.15 Z.1-2]{Hertzberg.2012}  % Angabe mit Seite+Zeile
\end{verbatim} } 
wird ein Zitat mit Verweis auf unsere Quelle angegeben.

\cite{Hertzberg.2012} \\
\cite[S. 15]{Hertzberg.2012} \\
\cite[S.15 Z.1-2]{Hertzberg.2012} \\

\section{Zitate - Quellen}

Wie bereits zuvor angedeutet verwenden Zitate eine .bib Datei, welche man leicht mit einer Quellenverwaltung erstellen kann. Hierf�r gibt es von der Hochschule Lizenzen f�r Citavi sowie Zotero. Da ich in meiner Arbeit Citavi+Texmaker verwende, gehe ich nur hierauf ein. Eine Einarbeitung in Zotero ist aber auch leicht.
{ \begin{verbatim} 
https://bib.hm.edu/tutorials/citavi/index.de.html
\end{verbatim} } 
Der Ablauf ist recht einfach:
\begin{enumerate}
  \item Projekt in Citavi erstellen.
  \item Per Citavi-Picker bzw manuell B�cher, Artikel etc hinzuf�gen.
  \item Citavi - Extras - Optionen - Zitation - Standardeditior auf TexMaker.
  \item Citavi - Datei - Exportieren - Exportieren - Alle Titel - BibTeX - In Zwischenablage.
  \item Jetzt in die Bibliethek.bib im Ordner die Zwischenablage einf�gen.
  \item In TexMaker - Optionen - Texmaker konfigurieren - Schnelles �bersetzen
  \item PdfLatex + Bib(la)tex + PdfLatex(2x)+PDF Anzeigen.
  \item Beim n�chsten "'Schnelles �bersetzen"' wird die .bib genutzt.
\end{enumerate}


\section{Fu\ss noten}
Fu\ss noten k\"onnen auch leicht eingef\"ugt werden. Hierzu einfach
\begin{verbatim}
	\footnote{Ich bin der Fu\ss notentext}
\end{verbatim}
verwenden. Die Fu\ss note \footnote{Ich bin der Fu\ss notentext} wird dann am jeweiligen Seitenende angezeigt.

\section{Listen}
Listen kann man auch ganz leicht erstellen. Hierbei kann zwischen Listen mit Punkten und mit f�hrenden Zahlen gew�hlt werden.

\begin{minipage}[t]{0.5\textwidth}
\begin{itemize}
  \item Ein erster Punkt
  \item Ein zweiter Punkt
\end{itemize}
\end{minipage}
\begin{minipage}[t]{0.5\textwidth}
\begin{enumerate}
  \item Ein erster Punkt
  \item Ein zweiter Punkt
\end{enumerate}
\end{minipage}
\\
\begin{verbatim}
\begin{itemize}			// "enumerate" in den {}Klammern bewirken f�hrende Zahlen
  \item Ein erster Punkt
  \item Ein zweiter Punkt
\end{itemize}
\end{verbatim}
Damit die Tabellen wie oben nebeneinander sind kann der Befehl "minipage" genutzt werden. Man erstellt somit sozusagen eine Seite in der Seite.
\begin{verbatim}
\begin{minipage}[t]{0.5\textwidth}	//2 Elementen -> 0,5 Seitenbreite; 3 -> 0,3
\begin{itemize}						
  \item Ein erster Punkt
  \item Ein zweiter Punkt
\end{itemize}
\end{minipage}
\end{verbatim}
Um Unterlisten zu erstellen einfach in der Liste eine neue Liste beginnen. LaTex macht dann den Rest.

\clearpage

\section{Tabellen}
Tabellen werden mithilfe des \& Zeichens und den Befehlen tabular und hline erstellt.

\begin{minipage}[t]{0.5\textwidth}
\begin{center}
\vspace{12mm}
\begin{tabular}{|c|c|c|}
\hline
 A & B & C \\
\hline
 1 & 2 & 3 \\
\hline 
 4 & 5 & 6 \\
\hline
\end{tabular}
\end{center}
\end{minipage}
\begin{minipage}[t]{0.5\textwidth}
\begin{verbatim}
\begin{tabular}{|c|c|c|}
\hline
 A & B & C \\
\hline
 1 & 2 & 3  \\
\hline 
 4 & 5 & 6 \\
\hline
\end{tabular}
\end{verbatim}
\end{minipage}

\section{Archivierung}
F�r die Archivierung sind alle Dateien der Arbeit (auch der Vortr�ge)
dem Betreuer zur Verf�gung zu stellen.  Weiterhin soll noch ein
\BibTeX-Eintrag der Arbeit erstellt werden (die Felder in eckigen
Klammern sind dabei auszuf�llen):
\begin{verbatim}
@mastersthesis{<Nachname des Autors><Jahr>,	
  type =         {<Art der Arbeit>},
  title =        ,
  school =       {Munich University of applied Science (HM)},
  author =       {<Nachname des Autors>, <Vorname des Autors>},
  annote =       {<Nachname des Betreuers>, <Vorname des Betreuers>},
  month =        {<Monat>},
  year =         {<Jahr>},
  key =          {<Mehrere Suchschl�ssel>}
}
\end{verbatim}

%\include{StateOfTheArt}
%\include{Problem}
%\include{HowTo}
%\include{Results}
%  Conclusions (Zusammenfassung):
\include{Zusammenfassung}
% List of figures (Abbildungsverzeichnis):
% Latex erstellt diese Liste automatisch samt Angaben
\listoffigures
% List of tables (Tabellenverzeichnis):
% Latex erstellt diese Liste automatisch samt Angaben
\listoftables
% Glossary (Glossar):
%\include{Glossary}

% List of formulae (Liste der Formelzeichen):
% \include{Formulae}

% Appendix (Anhänge), could have multiple chaper-files:
\appendix
\include{Anhaenge}

% Abbreviations (Abkürzungsverzeichnis):
\chapter{Abk\"urzungsverzeichniss}
\thispagestyle{empty}% no page number in chapter title page
Beispiel f\"ur eine Tabelle:


\begin{acronym}[Bash]
 \acro{KDE}{K Desktop Enviroment}
 \acro{SQL}{Structured Query Language}
 \acro{Bash}{Bourne-again shell}
 \acro{JDK}{Java Development Kit}
 \acro{VM}{Virtuelle Maschine}
\end{acronym}


% References (Literaturverzeichnis):
% a) Style (with numbers: use unsrt):
\bibliographystyle{alpha}
% b) The File:
\bibliography{Bibliothek}

\end{document} %*****************************************************
